\message{ !name(test.tex)}\documentclass{beamer}
\usetheme{Madrid}
\usepackage{palatino}
\usepackage{tikz}
\usepackage{graphicx}

\usepackage[T1]{fontenc}
\usepackage[utf8]{inputenc}
\usefonttheme[onlymath]{serif}



\date[]{}
\title[La recherche du meilleur coup.]{À la recherche du meilleur coup au jeu d'Awalé}
\author[Nils Lelorieux]{Nils Lelorieux}
\institute[]{numéro de candidat : 24296}

\begin{document}

\message{ !name(test.tex) !offset(-3) }


\begin{frame}
  \titlepage
\end{frame}
 
\begin{frame}
  \frametitle{Introduction}


  \begin{minipage}[b]{0.3\linewidth}
    \begin{figure}
      \centering
      \includegraphics[width=1.5\linewidth]{ressources/photo_joueur_awale.jpeg}
      \caption{Une partie d'Awalé}
      \label{fig:mon_image}
      %% \label{fig:1}
    \end{figure}
  \end{minipage}
  \hfill
  \begin{minipage}[b]{0.48\linewidth}
    \begin{figure}
      \centering
      \includegraphics[width=0.8\linewidth]{ressources/wole_soyinka.jpeg}
      \label{fig1}
      \caption{Photographie de Wole Soyinka}
    \end{figure}
  \end{minipage}
      
\end{frame}

\begin{frame}
  \frametitle{Un jeu ancien et important}
  \begin{figure}
    \centering
    \includegraphics[width=\linewidth]{ressources/ancien_plateau_jeu.png}
    \caption{Un ancien plateau de jeu découvert au Kenya}
  \end{figure}
\end{frame}


\begin{frame}
  \begin{block}{Problématique}
    L'étude théorique du jeu d'Awalé permet-elle de toujours jouer le meilleur coup, ou l'utilisation d'algorithmes est-elle indispensable ?
  \end{block}
  \frametitle{Vue d'ensemble}
  \tableofcontents
\end{frame}


\section{Présentation des règles du jeu et motivations}
\begin{frame}
  \frametitle{Règles du jeu}
  \begin{minipage}[t]{0.5\linewidth}
    \begin{itemize}
    \item<Règles du jeu>
    \item Deux joueurs s'affrontent.
    \item 4 graines dans les 12 trous divisé en 2 rangée de 6.
    \item Le premier joueur choisi un puit et sème les graines.
    \item Si le dernier puit semé a 2 ou 3 pierres, il les récupère et fait la même chose dans le puit précédent, ...
      \item Le gagnant est le joueur qui a le plus de pierres
    \end{itemize}
  \end{minipage}
  \hfill 
  \begin{minipage}[t]{0.49\linewidth}
    \begin{figure}
      \centering
      \includegraphics[width=1.05\linewidth]{ressources/p_debut.jpg}
      \caption{L'état du plateau au début de la partie}
    \end{figure}
    
  \end{minipage}
  
\end{frame}

\begin{frame}
  \frametitle{Déroulé d'une partie}
  \begin{minipage}[b]{0.45\linewidth}
    \begin{figure}
      \centering
      \includegraphics[width=1\linewidth]{ressources/diagramme1.png}
      \caption{2 coups joués par les 2 joueurs en début de partie}
    \end{figure}
  \end{minipage}
  \hfill
  \begin{minipage}[b]{0.45\linewidth}    
    \begin{figure}
      \centering
      \includegraphics[width=1\linewidth]{ressources/diagramme2.png}
      \caption{Position où le joueur 1 capture les pierres du joueur 2}
    \end{figure}
  \end{minipage}
\end{frame}

\begin{frame}
  \frametitle{Motivations}
  \begin{itemize}
  \item Le facteur d'embrachement est assez faible.
  \item Possibilité d'utiliser des algorithmes classiques de la théorie des jeux pour trouver le coup optimal
  \item Il y a moins de $10^{14}$ états possibles.
  \end{itemize}
\end{frame}

\section{Échec de l'approche théorique et des premiers algorithmes}
\begin{frame}
  \frametitle{Une approche théorique}
  \begin{figure}
    \centering
    \includegraphics[width=\linewidth]{ressources/determined_positions.png}
    \caption{Un position déterminée}
  \end{figure}
\end{frame}

\begin{frame}
  \frametitle{L'algorithme min-max}
  \begin{figure}
    \centering
    \includegraphics[width=0.6\linewidth]{ressources/explication_min_max.png}
    \caption{Un exemple de min max au jeu du morpion}
  \end{figure}
\end{frame}


\begin{frame}
  \frametitle{Problème du min-max classique}
  \begin{figure}
    \centering
    \includegraphics[width=0.8\linewidth]{ressources/min_max_heuristique.png}
    \caption{Un exemple de min max avec heuristique}
  \end{figure}
\end{frame}

\begin{frame}
  \frametitle{Résultat min-max}
  \begin{figure}
    \centering
    \includegraphics[width=0.75\linewidth]{ressources/resultat_min_max.png}
    \caption{Résultats de l'algorithme min-max (1000 parties)}
  \end{figure}
\end{frame}

\begin{frame}
  \frametitle{Un algorithme de Monte Carlo}
  \begin{figure}
    \centering
    \includegraphics[width=\linewidth]{ressources/monte_carlo_explication.png}
    \caption{Principe de l'algorithme de Monte Carlo}
  \end{figure}
\end{frame}

\begin{frame}
  \frametitle{Des résultats peu convaincants}
  \begin{figure}
    \centering
    \includegraphics[width=0.75\linewidth]{ressources/resultat_flat_monte_carlo.png}
    \caption{Résultat de l'algorithme flat Monte Carlo (1000 parties)}
  \end{figure}
\end{frame}

\section{Des résultats concluents grâce à deux nouvelles techniques : la recherche arborescente de Monte-Carlo et l'analyse rétrograde}
\begin{frame}
  \frametitle{Une amélioration de l'algorithme précédent}
  \begin{figure}
    \centering
    \includegraphics[width=\linewidth]{ressources/monte_carlo_explication.png}
    \caption{Principe de la recherche arborescente de Monte Carlo}
  \end{figure}
\end{frame}

\begin{frame}
  \frametitle{Comment choisir le noeud à explorer ?}
  \begin{block}{Score d'un noeud}
On attribue à chaque noeud $n$ un score $Q$ qui correspond à la moyenne des gains remportés part toutes les parties contenant le noeud $n$. Étant donné un noeud $n$ et son père $p$, on a :
$$ Q(n) = \frac 1 {N(n)} \sum_{i=1}^{N(p)}\mathbb I_i(n)z_i $$
où $N(n)$ représente le nombre de visites du noeud $n$, $z_i$ est le gain lors de la ième visite de noeud $n$ et $\mathbb I_i(n)$ vaut $1$ si le noeud $n$ a ete choisie à la $i$ visite par $p$, 0 sinon
  \end{block}
\end{frame}

\begin{frame}
  \frametitle{Comment choisir le noeud à explorer (2) ?}
  \begin{block}{Meilleur fils d'un noeud $n$}
    $$\underset{n' \in \text{ enfants de } n}{\arg\max} \frac {Q(n')} {N(n')} + c\sqrt{\frac {2ln(N(n))} {N(n')} }$$
  \end{block}
\end{frame}


\begin{frame}
  \frametitle{Visualisation des arbres}
   \begin{figure}
     \centering
     \includegraphics[width=\linewidth]{ressources/arbre_40_flat.pdf}
     \caption{Un arbre de recherche pour le flat Monte Carlo}
   \end{figure}
\end{frame}

\begin{frame}
  \frametitle{Visualisation des arbres}
  \begin{figure}
    \centering
    \includegraphics[width=\linewidth]{ressources/arbre_40_mcts.pdf}
    \caption{Un arbre de recherche pour la recherche arborescente de Monte Carlo}
  \end{figure}
\end{frame}

\begin{frame}
  \frametitle{Résultats de la recherche arborescente de Monte Carlo}
  \begin{figure}
    \centering
    \includegraphics[width=0.7\linewidth]{ressources/resultat_mcts.png}
    \caption{Résultats de la recherche arborescente de Monte Carlo}
  \end{figure}
\end{frame}

\begin{frame}
  \frametitle{Un nouveau paradigme : l'analyse rétrograde}
  \begin{itemize}
  \item Utilisation de la programmation dynamique
  \end{itemize}
  \begin{block}{Anti-coups}
    Étant donné une configuration à $n$ pierres $A$, on dit que $A'$ est obtenue par anti-coup depuis la configuration $A$ si il existe un coup jouable dans $A'$ menant à la configuration $A$ sans récolte
  \end{block}
\end{frame}

\begin{frame}
  \frametitle{Explication de l'analyse rétrograde}
  L'algorithme se déroule en 3 étapes :
  \begin{itemize}
  \item Initialisation
  \item Convergence
  \item Stabilisation
  \end{itemize}
  Mais comment stocker toutes ses positions ?
\end{frame}

\begin{frame}
  \frametitle{Codage de Gödel}
  \begin{block}{Vecteur $\vec c$}
    Étant donné une configuration $A$, on note $A(i)$ le nombre de pierre dans le puit $i$. On pose $\vec c = (c_0, ..., c_{11})$ où $c_0 = A(0)$ et $c_{k+1}=A(k+1) + 1 + c_k$
  \end{block}
  \begin{block}{Codage de Gödel }
    On note, $$enc(\vec c) = \sum_{i=1}^{12} \binom {c_{i-1}} {i}$$
    Et on appelle nombre de Gödel de $A$ la quantité
    $$\mathcal E(A) = n + 25 \times enc(\vec c)$$ où $n$ est le nombre de pierres de $A$
  \end{block}
  Base de donnée construite pour $n = 29$
\end{frame}

\begin{frame}
  \frametitle{Conclusion}
\end{frame}
  

\end{document}

\message{ !name(test.tex) !offset(-275) }
